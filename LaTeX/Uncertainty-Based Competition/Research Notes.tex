\documentclass[10pt]{article}

\usepackage[backend=biber, style=authoryear, sorting=ynt]{biblatex}
\addbibresource{Research.bib}

\title{Research Notes}
\author{Richard Maskell - 1238287}
\begin{document}
\maketitle
\date

\section{Uncertainty-Based Competition Between Prefrontal and Dorsolateral Striatal Systems for Behavioural Control}
	
	\subsection{Document Overview}

		This article by Daw, Niv \& Dayan is based on the idea that ``neural systems, notably prefrontal cortex, the striatum and their dopaminergic afferents, are thought to contribute to the selection of actions'' \parencite{Daw}

		When these systems disagree, the different neurological structures compete with each other, which is modelled here using a ``Bayesian principle of arbitration between them according to uncertainty, so each controller is deployed when it should be most accurate''. \parencite{Daw}


\subsection{The Dorsolateral Striatum}

	It is stated that ``the dorsolateral striatum and its dopaminergic afferents support habitual or reflexive control'' \parencite{Daw}

\subsection{The Prefrontal Cortex}
	
	The prefrontal cortex is said to be ``associated with more reflective or cognitive action planning'' \parencite{Daw} along with additional regions which are excluded for simplicity in this article.

\subsection{Outcome Re-valuation}
	
	Conditioning studies where the subject learns desirable behaviour through rewards and/or punishments can have their reward values unexpectedly changed in order to differentiate between two control systems. ``Outcome re-valuation affects the two styles of control differently and allows investigation of the characteristics of each controller, its neural substrates and the circumstances under which it dominates'' \parencite{Daw}

\subsection{Normative Questions}

	``Why should the brain use multiple action controllers
	How should action choice be determined when they disagree'' \parencite{Daw}


\subsection{Reinforcement Learning}

	``In reinforcement learning, candidate actions are assessed through predictions of their values, defined in terms of the amount of reward they are expected eventually to bring about'' \parencite{Daw}.

	\subsubsection{Deferred Rewards}

		Deferred rewards, such as those dependant on multiple consecutive actions, present complications in predicting the value of actions in reinforcement learning as an initial choice in the sequence may not produce any immediate rewards, only a deferred one. Two different classes of reinforcement learning are used to produce different action's values, which are model-free approaches and model-based approaches.

	\subsubsection{The Model-Free Approach}

		``[These] underpin existing popular accounts of the activity of dopamine neurons and their (notably dorsolateral) striatal projections'' \parencite{Daw}.

	\subsubsection{The Model-Based Approach}

		``[This approach] involves 'model-based' methods, which we identify with the prefrontal cortex system'' \parencite{Daw}

	\subsubsection{Accuracy of Different Reinforcement Learning Approaches}

		The different accuracy ratings of each approach is proposed by \textcite{Daw} to justify ``the plurality of control and [underpin] arbitration'' where the brain relies on the specific control system in the circumstances where it's predictions tend to be most accurate.

		Such accuracy is suggested by \textcite{Daw} to be estimated for the ``purpose of arbitration by tracking the relative uncertainty of the predictions made by each controller.''

		Strict separation between the systems is assumed in order to isolate their hypothesis

	



\subsection{Notes}

\subsection{Conclusions}

\printbibliography

\end{document}